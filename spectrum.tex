\documentclass[main]{subfiles}

\begin{document}
\section{Спектр и резольвента}
В этом параграфе $E$ "--- банахово пространство над $\Complex$,
$A \in \mathcal{L}(E)$ и $\lambda \in \Complex$.
Введём обозначение $A_\lambda \coloneqq A - \lambda I$.

Воспоминание: в линейной алгебре мы изучали собственные значения,

\begin{definition}
  $\lambda \in \Complex$ называется \emph{регулярной}
  ($\lambda \in \rho(A)$), если
  $\exists (A - \lambda I)^{-1} \in \mathcal{L}(E)$.
  Иначе, $\lambda$ принадлежит спектру ($\lambda \in \sigma(A)$).
\end{definition}

\begin{example}
  Если $\dim E < \infty$, то $\sigma(A)$ есть в точности
  множество собственных значений оператора $A$.
\end{example}

Если $\Ker A_\lambda \ne \{ 0 \}$, то оператор необратим
и $\lambda \in \sigma_P(A)$ (точечный спектр); пусть это не так.
Если $\Img A_\lambda = E$, то по теореме банаха $\lambda \in \rho(A)$.
Если это не так, но $\overline{\Im A_\lambda} = E$, то $\lambda \in
\sigma_C(A)$ (непрерывный спектр). Наконец, в противном случае
$\lambda \in \sigma_R(A)$ (остаточный спектр).

\begin{theorem}
  Пусть $E(\Complex)$ "--- БП, $A \in \mathcal{L}(E)$.
  Тогда $\sigma(A)$ "--- замкнутое непустое множество
  и, более того,
  \[
    r(A) \coloneqq \sup_{\lambda \in \sigma} |\lambda| =
    \lim_{n \to \infty} \sqrt[n]{||A^n||}.
  \]
\end{theorem}

$\mathcal{L}(E)$ "--- банахова алгебра.
>= 1972 В. М. Тихомиров

\begin{proposition}
  $|\lambda| > ||A|| \To \lambda \in \rho(A)$.
\end{proposition}
\begin{proof}
  $A - \lambda I = - \lambda (I - \frac{1}{\lambda} A)$,
  и тогда, т. к. $||-\frac{1}{\lambda} A|| < 1$,
  по теореме существует обратный $-\frac1\lambda \sum \frac1{\lambda^n} A^n$.
\end{proof}

Резольвентой назовём $R_\lambda \coloneqq A_\lambda^{-1}$.

\begin{proposition}
  $\rho(A)$ "--- открытое множество/
\end{proposition}
\begin{proof}
  Пусть $\lambda \in \rho(A)$, тогда
  если $||\delta A|| < \frac{1}{||A^{-1}||}$
\end{proof}

\begin{proposition}
  $R_\lambda$ непрерывна на $\rho(A)$
\end{proposition}
\begin{proof}
  Аналогично применяем теорему, точнее,
  \[ ||(A + \delta A)^{-1} - A^{-1}|| \le
    \frac{||A^{-1}||^2 ||\delta A||}{1 - ||\delta A|| \cdot ||A^{-1}||}
  \]
\end{proof}

\begin{proposition}[равенство Гильберта]
  Пусть $\lambda_0, \lambda \in \rho(A)$, тогда
  \[
    R_\lambda - R_{\lambda_0} = (\lambda - \lambda_0)
    R_\lambda R_{\lambda_0}.
  \]
\end{proposition}
\begin{proof}
  По определению, $R_\lambda A_\lambda = A_\lambda R_\lambda = I$,
  поэтому
  \[ R_\lambda - R_{\lambda_0} = 
    R_\lambda A_{\lambda_0} R_{\lambda_0} -
    R_\lambda A_\lambda R_{\lambda 0} =
    R_\lambda (A_{\lambda_0} - A_\lambda) R_{\lambda_0} =
    R_\lambda (\lambda E - \lambda_0 E) R_{\lambda_0} =
    (\lambda - \lambda_0) R_\lambda R_{\lambda_0}.
  \]
\end{proof}

\begin{proposition}
  $R_\lambda$ дифференцируема на $\rho(A)$.
\end{proposition}
\begin{proof}
  \[  \frac{R_\lambda - R_{\lambda_0}}{\lambda - \lambda_0} =
    \frac{(\lambda - \lambda_0) R_\lambda R_{\lambda_0}}{\lambda - \lambda_0} =
    R_\lambda R_{\lambda_0} \xrightarrow{\lambda \to \lambda_0} R_{\lambda_0}^2.
    \qedhere
  \]
\end{proof}

\begin{proposition}
  Радиус сходимости ряда равен $r(A)$.
\end{proposition}
\begin{proof}
  Пусть $|\lambda| > r(A)$, тогда $\lambda \in \rho(A)$
  вне круга $|\lambda| \le r(A)$.
  $R_\lambda$ разложим в ряд Лорана, который
  сходится при $|\lambda| > |r(A)|$.
  
  Пусть $|\lambda_| < r(A)$. Утверждается, что ряд Неймана
  расходится в этой точке. Пусть это не так, тогда
  он сходится и для $|\lambda| > |\lambda_0|$,
  и тогда $r(A)$ "<не является спектральным радиусом">.
\end{proof}

\begin{proposition}
  $\lambda \in \sigma(A) \To \lambda^n \in \sigma(A^n)$.
\end{proposition}

\begin{exercise}
  $\sigma(A^n) = (\sigma(A))^n $.
\end{exercise}

\begin{proof}
  Предоложим противное: $\lambda^n \in \rho(A^n)$, т. е.
  $(A - \lambda I)^{-1} \in \mathcal{L}(E)$.
  Тогда умножим равенство
  \[ (A^n - \lambda^n I) = (A - \lambda I)
  (A^{n -1} + \lambda A^{n-2} + \dots + \lambda^{n-1} I) \]
  справа на $(A^n - \lambda^n I)^{-1}$ мы покажем, что
  $(A^{n -1} + \lambda A^{n-2} + \dots + \lambda^{n-1} I)(A^n - \lambda^n I)^{-1}$
  "--- правый обратный для $(A - \lambda I)$, аналогично находим левый обратный,
  и тогда $\lambda \in \rho(A)$.
\end{proof}

\begin{proposition}
  $\sigma(A) \ne \emptyset$
\end{proposition}
\begin{proof}
  Предположи противное, т. е. что $R_\lambda$ "--- целая функция.
  А там тип теорема Лиувилля.
\end{proof}

\begin{proposition} % 6
  $ r(A) = \lim \sqrt[n]{||A^n||} $.
\end{proposition}
\begin{proof}
  Имеем: $r(A)  = r_{\text{сх}} = \limsup \sqrt[n]{||A^n||}$.

  По утв $(r(A))^n \le r(A^n)$, или
  $r(A) \le \left(r(A^n) \right)^{1/n} \le ||A^n||^{1/n}$.
  Значит,
  \[ \liminf_{n \to \infty} \sqrt[n]{||A^n||} \ge r(A) = \limsup_{n \to \infty} \sqrt[n]{||A^n||}, \]
  а тогда верхний и нижний пределы совпадают, и существует равный им предел последовательности.
\end{proof}
 
\end{document}
