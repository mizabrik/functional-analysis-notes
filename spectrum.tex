\documentclass[main]{subfiles}

\begin{document}
\section{Спектр и резольвента} % 7
Воспоминание: в линейной алгебре
мы изучали собственные значения линейных операторов,
т. е. такие числа \( \lambda \),
что для некоторого ненулевого \( x \)
\[ Ax = \lambda x; \]
очевидно, это эквивалентно выполнению равенства
\[ \det (A - \lambda I) = 0. \]

% связь с ОТА, её доказательство в ТФКП

Банахова алгебра: +умножение, асс. дистрибутивно, единица,
\( ||xy|| \le ||x|| \cdot ||y|| \), \( ||e|| = 1 \).

Далее мы будем пользоваться тем фактом, что многие результаты
курса комплексного можно перенести на отображения из
комплексных чисел в некоторую банахову алгебру.
Понятия предела, производной и

Далее в этом параграфе,
если не указано обратное,
\( E \) "--- банахово пространство над \( \Complex \),
$A \in \Linears{E}$ и $\lambda \in \Complex$.
Также введём обозначение $A_\lambda \coloneqq A - \lambda I$.

Самое важная для приложений — существование
ОНБ из с. в. для эрмитового оператора.
Например, для использования метода Фурье решения ДУ.
Функция Грина для перехода в интегральные, потом Гильберт-Шмидт,

\begin{theorem*}[Гильберт--Шмидт]
  Пусть \( H \) "--- сепарабельное гильбертово пространство,
  \( A \in \Linears{H} \) "--- компактный самосопряжённый оператор.
  Тогда в \( H \) найдётся ортонормированный базис,
  состоящий из состоящий из собственных векторов \( A \).
\end{theorem*}

\begin{definition}
  Пусть \( E \) "--- ЛНП над \( \Complex \),
  \( A \in \Linears{E} \).
  Точка \( \lambda \in \Complex \) называется \emph{регулярной}
  ($\lambda \in \rho(A)$), если
  $\exists (A - \lambda I)^{-1} \in \mathcal{L}(E)$.
  Иначе, $\lambda$ принадлежит спектру ($\lambda \in \sigma(A)$).
\end{definition}

\begin{example}
  Если $\dim E < \infty$, то $\sigma(A)$ есть в точности
  множество собственных значений оператора $A$.
\end{example}

Если $\Ker A_\lambda \ne \{ 0 \}$, то оператор необратим
и $\lambda \in \sigma_P(A)$ (точечный спектр); пусть это не так.
Если $\Img A_\lambda = E$, то по теореме банаха $\lambda \in \rho(A)$.
Если это не так, но $\overline{\Im A_\lambda} = E$, то $\lambda \in
\sigma_C(A)$ (непрерывный спектр). Наконец, в противном случае
$\lambda \in \sigma_R(A)$ (остаточный спектр).

\begin{theorem}
  Пусть $E(\Complex)$ "--- БП, $A \in \mathcal{L}(E)$.
  Тогда $\sigma(A)$ "--- замкнутое непустое множество
  и, более того,
  \[
    r(A) \coloneqq \sup_{\lambda \in \sigma} |\lambda| =
    \lim_{n \to \infty} \sqrt[n]{||A^n||}.
  \]
\end{theorem}

$\mathcal{L}(E)$ "--- банахова алгебра.
>= 1972 В. М. Тихомиров

\begin{proposition} % 0
  $\rho(A)$ "--- открытое множество.
\end{proposition}
\begin{proof}
  Заметим, что
  \[
    A - \lambda I = -\lambda (I - \frac{1}{\lambda} A
  \].
  Пусть $\lambda \in \rho(A)$, тогда
  если $||\delta A|| < \frac{1}{||A^{-1}||}$
\end{proof}

%\begin{proposition}
%  $|\lambda| > ||A|| \To \lambda \in \rho(A)$.
%\end{proposition}
%\begin{proof}
%  $A - \lambda I = - \lambda (I - \frac{1}{\lambda} A)$,
%  и тогда, т. к. $||-\frac{1}{\lambda} A|| < 1$,
%  по теореме существует обратный $-\frac1\lambda \sum \frac1{\lambda^n} A^n$.
%\end{proof}
%
Резольвентой назовём $R_\lambda \coloneqq A_\lambda^{-1}$.


\begin{proposition} % 1
  $R_\lambda$ "--- непрерывное отображение на $\rho(A)$.
\end{proposition}
\begin{proof}
  Благодаря ряду Неймана мы можем оценить изменение нормы:
  \( A + \Delta = A \cdot (I + A^{-1} \Delta) \),
  а потому
  \[
    (A + \Delta)^{-1} = \sum_{n=0}^\infty (-1)^n (A^{-1} \Delta)^n A^{-1},
  \]
  значт,
  \[
    ||(A + \Delta)^{-1} - A^{-1}|| \le \frac{||A^{-1}||^2 ||\Delta||}
    {1 - ||A^{-1}|| \cdot ||\Delta||}.
  \] %???
  Тогда при \( \lambda \to \lambda_0 \)
  \[
    ||R_{\lambda} - R_{\lambda_0}|| \le
    \frac{||R_{\lambda_0}||^2 \cdot |\lambda - \lambda_0|}
    {1 - ||R_{\lambda_0}|| \cdot |\lambda - \lambda_0|} \to 0.
  \]
\end{proof}

\begin{proposition}[равенство Гильберта] % 2
  Пусть $\lambda_0, \lambda \in \rho(A)$, тогда
  \[
    R_\lambda - R_{\lambda_0} = (\lambda - \lambda_0)
    R_\lambda R_{\lambda_0}.
  \]
\end{proposition}
\begin{proof}
  Интуицию нужно искать в приведении дробей к общему знаменателю.
  Если вместо операторов рассматривать числа, то всё очевидно:
  \[
    \frac1{A - \lambda \cdot 1} - \frac{1}{A - \lambda_0 \cdot 1} =
    \frac{(A - \lambda_0) - (A - \lambda)}{(A - \lambda)(A - \lambda_0)} =
    \frac{\lambda - \lambda_0}{(A - \lambda)(A - \lambda_0)}.
  \]
  Для случая операторов нужно вспомнить, что, по определению,
  $R_\lambda A_\lambda = A_\lambda R_\lambda = I$,
  а потому
  \[ R_\lambda - R_{\lambda_0} = 
    R_\lambda A_{\lambda_0} R_{\lambda_0} -
    R_\lambda A_\lambda R_{\lambda 0} =
    R_\lambda (A_{\lambda_0} - A_\lambda) R_{\lambda_0} =
    R_\lambda (\lambda I - \lambda_0 I) R_{\lambda_0} =
    (\lambda - \lambda_0) R_\lambda R_{\lambda_0}.
  \]
\end{proof}

\begin{proposition} % 3
  $R_\lambda$ дифференцируема на $\rho(A)$.
\end{proposition}
\begin{proof}
  С использованием предыдущих утверждений доказательство тривиально:
  \[
    \lim_{\lambda \to \lambda_0}
    \frac{R_\lambda - R_{\lambda_0}}{\lambda - \lambda_0} =
    \lim_{\lambda \to \lambda_0}
    \frac{(\lambda - \lambda_0) R_\lambda R_{\lambda_0}}{\lambda - \lambda_0} =
    \lim_{\lambda \to \lambda_0} R_\lambda R_{\lambda_0} = R_{\lambda_0}^2.
    \qedhere
  \]
\end{proof}

Ряд Неймана
\[
  -\frac{1}{\lambda} \sum_{n=0}^\infty \frac{1}{\lambda^n} A^n.
\]
Ряд Лорана.

\begin{proposition} % 4
  Радиус сходимости ряда равен спектральному радиусу $r(A)$.
\end{proposition}
\begin{proof}
  В кольце (в смысле комплексного анализа)
  \( \{ \lambda : |\lambda| > r(A) \} \)
  \( R_\lambda \) дифференциируема, а потому разложима в ряд Лорана.
  Кроме того, ряд, конечно, сходится для $|\lambda| > r(A)$,
  и единственность и всё ок.
  
  Пусть $|\lambda_| < r(A)$. Утверждается, что ряд Неймана
  расходится в этой точке. Пусть это не так, тогда
  он сходится и для $|\lambda| > |\lambda_0|$,
  и тогда $r(A)$ "<не является спектральным радиусом">.
\end{proof}


\begin{proposition} % 5
  \( \sigma(A) \ne \emptyset \).
\end{proposition}
\begin{proof}
  Предположи противное, тогда \( \rho(A) = \Complex \)
  и \( R_\lambda \) "--- целая функция.
  В то же время, при больших \( \lambda \)
  \[
    ||R_\lambda|| =
    ||-\frac{1}{\lambda} \sum_{n=0}^\infty \frac{1}{\lambda^n} A^n|| =
    \frac{1}{\lambda} \cdot \frac{1}{1 - \frac{||A||}{|\lambda|}} \to 0.
  \]
  Значит, \( R_\lambda \) также ограничена, и тогда по теореме Лиувилля
  она константа; кроме того, исходя из оценок на наорму,
  она тождественно равна нулю; это, конечно, невозможно для обратного оператора.
\end{proof}

\begin{proposition} % 6
  $\lambda \in \sigma(A) \To \lambda^n \in \sigma(A^n)$.
\end{proposition}

\begin{proof}
  Предоложим противное: $\lambda^n \in \rho(A^n)$, т. е.
  $(A^n - \lambda^n I)^{-1} \in \Linears{E}$.
  Тогда умножив равенство
  \[
    (A^n - \lambda^n I) = (A - \lambda I)
    (A^{n -1} + \lambda A^{n-2} + \dots + \lambda^{n-1} I)
  \]
  справа на \( (A^n - \lambda^n I)^{-1} \) мы покажем,
  что \( (A^{n -1} + \dots + \lambda^{n-1} I)(A^n - \lambda^n I)^{-1} \)
  "--- правый обратный оператор для \( (A - \lambda I) \).
  Повторив действия симметрично мы получим и левый обратный оператор;
  значит, по общим алгебраическим соображениям, они совпадают и
  являются обратным оператором.
  Но тогда \( \lambda \in \rho(A) \) "--- противоречие.
\end{proof}

\begin{exercise}
  \( \sigma(A^n) = (\sigma(A))^n \).
\end{exercise}

\begin{proposition}
  $ r(A) = \lim \sqrt[n]{||A^n||} $.
\end{proposition}
\begin{proof}
  Мы уже показали, что спектральный радиус равен радиусу сходимости
  ряда Лорана для \( R_\lambda \) на бесконечности;
  с другой стороны, радиусом сходимости этого ряда определяется
  формулой Коши-Адамара, а потому
  \[
    r(A) = \limsup_{n \to \infty} \sqrt[n]{||A^n||};
  \]
  теперь достаточно показать, что у этой последовательности есть предел.

  По утв $(r(A))^n \le r(A^n)$, или
  $r(A) \le \left(r(A^n) \right)^{1/n} \le ||A^n||^{1/n}$.
  Значит,
  \[ \liminf_{n \to \infty} \sqrt[n]{||A^n||} \ge r(A) = \limsup_{n \to \infty} \sqrt[n]{||A^n||}, \]
  а тогда верхний и нижний пределы совпадают, и существует равный им предел последовательности.
\end{proof}

\begin{exercise}
  Оператор Вольтерра в \( C[0, 1] \) определяется так:
  \[ (Af)(x) = \int_{0}^x f(t) dt. \]
  Докажите, что \( r(A) = 0 \), \( \sigma(A) = \{ 0 \} \)
  и при этом \( 0 \in \sigma_R(A) \).
\end{exercise}
 
\end{document}
