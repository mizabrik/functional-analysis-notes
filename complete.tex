\documentclass[main]{subfiles}

\begin{document}

\section{Полные метрические пространства}

\begin{definition}
  Пусть \( (X, \rho) \) "--- метрическое пространство,
  тогда последовательность \( \{ x_n \} \subset X \)
  называется \emph{фундаментальной},
  если
  \[
    \Forall{\epsilon > 0} \Exists{N} \Forall{n, m \ge N}
    \rho(x_n, x_m) < \epsilon.
  \]
\end{definition}

\begin{definition}
  Если в метрическом пространстве \( (X, \rho) \)
  любая фундаментальная последовательность имеет предел,
  оно называется \emph{полным}.
\end{definition}

\begin{exercise}
  Если \( X \) полно, то \( M \subset X \)
  замкнуто \( \oTTo \) \( M \) полно как подпространство.
\end{exercise}

\begin{theorem}[принцип вложенных шаров]\label{thm:complete-balls}
  Пусть \( (X, \rho) \) "--- полное метрическое пространство,
  \( \{ \Cl{B}_n \} \) "---
  последовательность вложенных замкнутых шаров с радиусами
  \( r_n \to 0 \). Тогда
  \[
    \bigcap_{n = 1}^\infty \Cl{B}_n = \{ x \}.
  \]
\end{theorem}
\begin{remark}
  Данное свойство эквивалентно полноте метрического пространства,
  т. е. характеризует её.
\end{remark}
\begin{remark}
  Без требования \( r_n \to 0 \), общей точки может не быть.
  Например, определим на \( \Natural \)
  метрику
  \[
    \rho(n, m) = \begin{cases}
      0, & n = m \\
      1 + \frac{1}{n + m}, & n \ne m
    \end{cases}.
  \]
  Полученное метрическое пространство полно
  (т. к. любая фундаментальная последовательность стационарна),
  но пересечение последовательность вложенных шаров
  \[
    \Cl{B}(n, 1 + \frac{1}{2n}) = \{ m \in \Natural \mid m \ge n \}
  \]
  пусто.
\end{remark}

\begin{proof}
  Пусть \( \{ x_n \} \) "--- последовательность центров
  шаров. Покажем, что она фундаментальна. Зафиксируем
  \( \epsilon > 0 \). Т. к. \( r_n \to 0 \),
  для некоторого \( N \) \( r_N < \epsilon \),
  и тогда для \( n, m \ge N \),
  т. к. \( \Cl{B}_n, \Cl{B}_m \subset \Cl{B}_N \),
  \[
    \rho(x_n, x_m) \le \rho(x_N, x_n) + \rho(x_N, x_m)
    < 2 \epsilon,
  \]
  а потому \( \{ x_n \} \) "--- действительно фундаментальна.

  Пользуясь полнотой выберем \( x = \lim x_n \).
  Благодаря вложенности, для \( m \ge n \)
  \( \rho(x_n, x_m) \le r_n \), и переходя
  к пределу по \( m \) в этом неравенстве мы
  получаем \( \rho(x_n, x) \le r_n \),
  т. е. \( x \in \Cl{B}_n \) для произвольного \( n \).
  Значит,
  \[
    x \in \bigcap_{n = 1}^\infty \Cl{B}_n.
  \]

  Осталось показать единственность. Если \( x \), \( y \) "---
  точки из пересечения шаров, то, т. к. \( x, y \in \Cl{B}_n \),
  \( \rho(x, y) \le \rho(x, x_n) + \rho(y, x_n) \le 2 \epsilon \to 0 \),
  откуда \( \rho(x, y) = 0 \) и \( x = y \).
\end{proof}

\begin{exercise}
  Доказать теорему для последовательности
  вложенных замкнутых множеств \( \{ F_n \} \)
  таких, что
  \[
    \sup_{x, y \in F_n} \rho(x, y) \to 0, \quad n \to \infty.
  \]
\end{exercise}

\begin{theorem}[Бэр, б/д]
  %Пусть \( X \) "--- полное метрическое пространство.
  %Тогда \( X \) нельзя представить в виде
  Полное метрическое пространство
  нельзя представить в виде счётного объединения
  нигде не плотных множеств, т. е.
  оно ялвяется множеством второй категории
\end{theorem}

\begin{example}
  \( C[a, b] \) "--- множество второй категории,
  но в нём можно выделить множество первой
  категории (т. е. представимое в виде счётного
  объединения нигде не плотных множеств):
  \[
    \{ f \in C[a, b] \mid \Exists{x \in [a, b]}
      \text{в \( f(x) \) существует односторонняя производная}
    \}.
  \]
\end{example}

\begin{definition}
  Пусть \( (X, \rho) \) "--- метрическое пространство, 
  тогда отображение \( f : X \to X \) называется \emph{сжимающим},
  если \( \Forall{x, y \in X} \rho(f(x), f(y)) \le \alpha \rho(x, y) \)
  для некоторого \( \alpha \in (0, 1) \).
\end{definition}

\begin{theorem}[Банах, 1922, принцип сжимающих отображений]
  Пусть \( X \) "--- полное метрическое пространство,
  \( f : X \to X \) "--- сжимающее отображение,
  тогда \( \ExistsOne{x \in X} f(x) = x \).
\end{theorem}
\begin{proof}
  Выберем произвольный \( x_0 \in X \) и для
  для \( n = 0, \dots \) положим \( x_{n + 1} = f(x_n) \);
  покажем фундаментальность \( \{ x_n \} \).
  Индукцией легко показать, что
  \[
    \rho(x_n, x_{n+1}) \le \alpha^n \rho(x_0, x_1),
  \]
  а тогда для произвольного \( p > 0 \)
  \[
    \rho(x_n, x_{n+p}) \le
    \sum_{k = n}^{n + p - 1} \rho(x_k, x_{k + 1}) \le
    \sum_{k = n}^{n + p - 1} \alpha^k \rho(x_0, x_1) \le
    \sum_{k = n}^\infty \alpha^k \rho(x_0, x_1) =
    \frac{\alpha^n}{1 - \alpha} \rho(x_0, x_1) \to 0.
  \]
  Итак, \( \{ x_n \} \) фундаментальна и мы можем выбрать
  \( x = \lim x_n \) благодаря полноте \( X \).

  Заметим: \( f \) "--- непрерывна, т. к.
  если \( y_n \to y \), то
  \[
    \rho(f(y_n), f(y)) \le \alpha \rho(y_n, y) \to 0,
  \]
  откуда \( \rho(f(y_n), f(y)) \to 0 \), или
  \( f(y_n) \to f(y) \).
  А значит, мы можем совершить предельный переход
  в равентсве \( x_{n + 1} = f(x_n) \), получая
  \( x = f(x) \).

  Осталось доказать единственность неподвижной точки.
  Если \( x \), \( y \) "--- неподвижные точки \( f \),
  то
  \[
    \rho(x, y) = \rho(f(x), f(y)) \le \alpha \rho(x, y),
  \]
  что возможно только при \( \rho(x, y) = 0 \),
  ведь \( \alpha < 1 \). Ну а это означает, что \( x = y \).
\end{proof}

\begin{exercise}
  Проверьте, всегда ли неподвижная точка будет существовать,
  если \( f \) не сжимающее, но для \( x \ne y \)
  \( \rho(f(x), f(y)) < \rho(x, y) \). А если \( X \) компактно?
\end{exercise}

\begin{theorem}[Хаусдорф, б/д]
  Для любого неполного МП \( X \) существует и единственно его
  пополнение \( Y \), т. е. полное метрическое пространство,
  в котором существует всюду плотное подмножество, изометричное \( X \).
\end{theorem}

\begin{example}
  \( \Real \) "--- пополнение \( \Rational \).
\end{example}

\begin{exercise}
  Доказать теорему Хаусдорфа. Подсказка: ввести
  отношение эквивалентности на классе всех
  фундаментальных последовательностей,
  где \( \{ x_n \} \sim \{ y_n \} \),
  если \( \rho(x_n, y_n) \to 0 \),
  после чего рассмотреть фактормножество по \( \sim \)
  с метрикой
  \( \rho([\{ x_n \}], [\{ y_n \}]) = \lim \rho(x_n, y_n) \).
\end{exercise}

\end{document}
