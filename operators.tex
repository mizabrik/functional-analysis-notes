\documentclass[main]{subfiles}

\begin{document}
\section{Линейные ограниченные операторы}

\begin{definition}
  Пусть \( E_1 \), \( E_2 \) "--- ЛНП,
  тогда оператор \( A : E_1 \to E_2 \) называется
  \emph{ограниченным}, если он переводит ограниченные
  множества в ограниченные. \( A \) называется
  \emph{линейным}, если \( A(\alpha_1 x_1 + \alpha_2 x_2)
  = \alpha_1 A(x_1) + \alpha_2 A(x_2) \).
\end{definition}

\begin{proposition}
  Пусть \( E_1 \), \( E_2 \) "--- ЛНП, \( A : E_1 -> E_2 \) "---
  линейный оператор, непрерывный в т. \( x_0 \).
  Тогда \( A \) "--- непрерывен.
\end{proposition}
\begin{proof}
  Пусть \( x_n \to x \), тогда \( x_n - x + x_0 \to x_0 \To
  A(x_n - x + x_0) = A(x_n) - A(x) + A(x_0) \to A(x_0) \To
  A(x_n) \to A(x) \).
\end{proof}

\begin{proposition}
  Пусть \( E_1 \), \( E_2 \) "--- ЛНП, \( A : E_1 -> E_2 \) "---
  линейный оператор.
  Тогда \( A \) ограничен \( \oTTo \) \( A(\overline{B}(0; 1)) \) "---
  ограниченно.
\end{proposition}

\begin{lemma}
  \[ ||A|| = \sup_{||x|| < 1} ||A(x)|| = \sup_{||x|| = 1} ||A(x)|| =
  \sup_{x \ne 0} \frac{||A(x)||}{||x||}. \]
\end{lemma}

\begin{theorem}
  Пусть \( E_1 \), \( E_2 \) "--- ЛНП, \( A : E_1 -> E_2 \) "---
  линейный оператор. Тогда \( A \) непрерывен \( \oTTo \)
  \( A \) ограничен.
\end{theorem}
\begin{itemproof}
  \item[\(\oT\)] Пусть \( x_n \to x_0 \), т. е. \( ||x_n - x_0|| \to 0 \).
    Тогда \( ||A(x_n) - A(x_0)|| = ||A(x_n - x_0)|| \le ||A|| ||x_n - x_0||
    \to 0 \), т. е. \( A(x_n) \to A(x_0) \).
  \item[$\To$]
    Предположим противное. Тогда \( \Forall{n} \Exists{x_n} ||x_n|| = 1,
    ||A(x_n)|| > n \). Рассмотрим \( y_n = \frac{x_n}{n} \):
    \( y_n \to 0 \), но \( ||A(y_n)|| = \frac{||A(x_n)||}{n} = 1 \),
    а потому \( A(y_n) \not \to A(0) = 0 \).
\end{itemproof}

\begin{theorem}
  Пусть \( E_1 \), \( E_2 \) "--- ЛНП.
  Обозначим через \( \mathcal{L}(E_1, E_2) \)
  множество линейно ограниченных операторов.
  Это линейное пространство, если определить
  \( (\alpha_1 A_1 + \alpha_2 A_2)(x) \coloneqq
  \alpha_1 A_1 (x_1) + \alpha_2 A_2(x_2) \),
  и норма оператора "--- норма в этом пространстве.
  Если \( E_2 \) "--- банахово, то
  \( \mathcal{L}(E_1, E_2) \) "--- также банахово.
\end{theorem}
\begin{proof}
  Линейность "--- очевидно, единтсвенный нетривиальный
  момент в проверке корректности нормы "--- неравенство
  треугольника.
  \[
    \sup_{||x|| \le 1} ||(A_1 + A_2)(x)|| =
    \sup_{||x|| \le 1} ||A_1(x) + A_2(x)|| \le
    \sup_{||x|| \le 1} ||A_1(x)|| + ||A_2(x)|| \le
    \sup_{||x|| \le 1} ||A_1(x)|| + \sup_{||x|| \le 1} ||A_2(x)||,
  \]
  т. е. \( ||A_1 + A_2|| \le ||A_1|| + ||A_2|| \).

  Если \( ||A_n - A|| \to 0 \), то
  \( \Forall{x \in E_1} ||A_n(x) - A(x)|| \to 0 \) "---
  поточечная сходимость следует из сходимости по норме
  (обратное, вообще говоря, неверно).
  Докажем полноту по такой схеме: для произвольной
  фундаментальной последовательности найдём её поточечный
  предел, а потом докажем что это линейный ограниченный оператор
  и что сходимость выполняется и по норме.
  
  Пусть \( \{ A_n \} \) "--- фундаментальная последовательность:
  \( \Forall{\epsilon} \Exists{N} \Forall{n, m \ge N} ||A_n - A_m|| < \epsilon \).
  Тогда \( \Forall{x \in E_1} ||A_n(x) - A_m(x)|| \le
  ||A_n - A_m|| ||x|| < \epsilon ||x|| \). Т. к.
  \( ||x|| \) "--- константа, то \( \{ A(x_n) \} \) "--- фундаментальна
  в \( L_2 \), а потому сходится. Положим
  \[ A(x) = \lim_{n \to \infty} A_n(x). \]

  Покажем, что \( A \) "--- линейный ограниченный оператор.
  \[
    A(\alpha_1 x_1 + \alpha_2 x_2) \ot A_n(\alpha_1 x_1 +
    \alpha_2 x_2) = \alpha_1 A_n(x_1) + \alpha_2 A_n(x_2)
    \to \alpha_1 A(x_1) + \alpha_2 A(x_2),
  \]
  и, т. к. предел единственнен, \( A \) "--- линейный оператор.
  Раз \( \{ A_n \} \) фундаментальная, то она ограниченна,
  т. е. \( \Exists{M} \Forall{n} ||A_n|| \le M \).
  Тогда \( ||A_n(x)|| \le ||A_n||||x|| \le M ||x|| \),
  и из предельного перехода \( ||A(x)|| \le M ||x|| \) "---
  \( A \) ограничен.

  Покажем, что \( ||A_n - A|| \to 0 \).
  \( \Forall{\epsilon > 0} \Exists{N} \Forall{n, m \ge N}
  ||A_n - A_m|| < \epsilon \).
  Если \( ||A_n - A_m|| < \epsilon \), то для произвольного
  \( x \in E_1 \) \( ||(A_n - A_m)(x)|| \le \epsilon ||x|| \),
  и после предельного перехода по \( m \) \( ||(A_n - A)(x)|| \le \epsilon ||x|| \).
  Отсюда следует, что \( ||A_n - A|| \le \epsilon \),
  т. е. \( \Forall{\epsilon > 0} \Exists{N}
  \Forall{n \ge n} ||A_n - A|| \le \epsilon \) "---
  \( ||A_n - A|| \to 0 \).
\end{proof}

\begin{corollary}
  Два важных частных случая:
  \begin{enumerate}
    \item \( E \) "--- банахово пространство, тогда \( \mathcal{L}(E) \) "---
      банахово.
    \item \( E \) "--- НП над \( F \) (это \( \Real \) или \( \Complex \)),
      тогда \( E^* = \mathcal{L}(E, F) \) "--- банахово.
  \end{enumerate}
\end{corollary}

\begin{theorem}
  Пусть \( E_1 \), \( E_2 \) "--- ЛНП.

  Тогда  \( \ExistsOne{\widetilde{A} \in \mathcal{L}(E_1, E_2)}
  \widetilde{A} \bigr{|}_{D(A)} = A, ||\widetilde{A}|| = ||A|| \).
\end{theorem}
\begin{proof}
  Предположим, что у нас есть продолжение
  \( \widetilde{A} \in \mathcal{L}(E_1, E_2) \). Тогда
  \( \Forall{x \in \overline{D(A)} = E_1} \)
\end{proof}

\end{document}
