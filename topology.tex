\documentclass[main]{subfiles}

\begin{document}

\subsection{Топологические пространства}

\begin{definition}
  Назовём пару \( (X, \tau) \), где \( \tau \subset 2^X \),
  \emph{топологическим пространством}
  (а \( \tau \) "--- \emph{топологией}),
  если выполнены следующие условия:
  \begin{enumerate}
    \item \( \emptyset, X \in \tau \)
    \item \( \bigcup_\alpha G_\alpha \in \tau \), если
      \( \Forall{\alpha} G_\alpha \in \tau \)
    \item \( \bigcap_{k = 1}^n G_k \in \tau \), если
      \( \Forall{k \in \{ 1, \dots, n \}} G_k \in \tau \)
  \end{enumerate}
  Элементы \( \tau \) называются открытыми множествами,
  а их дополнения "--- замкнутыми.
\end{definition}

Далее в этом параграфе будем считать,
что \( (X, \tau) \) "--- топологическое пространство,
если не оговорено иного.

\begin{definition}
  Пусть \( Y \subset X \), тогда топология
  \[
    \tau_Y = \{ G \cap Y \mid G \in \tau \}
  \]
  называется \emph{индуцированной} \( \tau \),
  а \( (Y, \tau_Y) \) "--- \emph{подпространством} \( (X, \tau) \).
\end{definition}

\begin{definition}
  \emph{Окрестностью} \( x \in X \) назовём произвольное
  открытое множество \( U \in \tau \) такое,
  что \( x \in U \), обозначение "--- \( U(x) \).
\end{definition}

\begin{definition}
  Если \( x \in X \) и \( M \subset X \), то \( x \)
  называется
  \begin{itemize}
    \item \emph{точкой прикосновения} \( M \), если
      \( \Forall{U(x)} M \cap U \ne \emptyset \)
    \item \emph{предельной точкой} \( M \), если
      \( \Forall{U(x)} \Exists{m \in M \setminus \{ x \}} m \in U(x) \).
    \item \emph{внутренней точкой} \( M \), если
      \( \Exists{U(x)} U(x) \subset M \).
  \end{itemize}
  \emph{Замыканием} \( M \) называется множество его точек прикосновения
  \( \overline{M} \), а \emph{внутренностью} "--- множество его внутренних
  точек \( \Int M \).
\end{definition}

\begin{definition}
  Последовательность \( x_n \) сходится к \( x \), если
  \( \Forall{U(x)} \Exists{N} \Forall{n \ge N} x_n \in U \).
\end{definition}

\begin{exercise}
  Доказать эквивалентность последних двух определений аналогичным
  метрическим, если в качестве топологии взято семейство
  открытых множеств в смысле метрического пространства,
  а также что \( G = \Int G \) и \( F = \Cl F  \)
\end{exercise}

\begin{remark}
  Как и в метрических пространствах,
  из существовании последовательности
  \( \{ x_n \} \subset M \) т. ч. \( x_n \to x \)
  следует, что \( x \) "--- точка прикосновения \( M \).
  Обратное, вообще говоря, не верно.
\end{remark}

\begin{definition}
  \( A \) плотно в \( B \), если \( B \subset \overline{A} \).
\end{definition}

\begin{exercise}
  Можно ли заменить в определение нигде не плотного множества в МП
  шар на произвольную окрестность? Как быть в ТП?
\end{exercise}

\begin{example}
  Для произвольного множества \( X \) можно ввести две
  топологии: антидискретная \( \{ \emptyset, X \} \)
  и дискретная \( 2^X \).
\end{example}

\begin{definition}
  Топология \( \tau_2 \) \emph{сильнее} топологии \( \tau_1 \)
  (\( \tau_1 \preceq \tau_2 \)),
  если \( \Forall{M \in \tau_1} M \in \tau_2 \),
  т. е. \( \tau_1 \subset \tau_2 \).
\end{definition}

\begin{remark}
  Вообще говоря, две различных топологии могут быть несравнимы.
\end{remark}

\begin{example}
  На \( X = C[a, b] \) \( \rho_1(f, g) = \int_a^b |f(x) - g(x)| dx \),
  порождает топологию сильнее, чем
  \( \rho_2(f, g) = \max_x |f(x) - g(x)| \).
\end{example}

\begin{remark}
  Обычно, если топология метризуема,
  то она считается достаточно сильной.
  Например, в МП \( \Forall{x \ne y}
  \Exists{U(x), V(y)} U \cap V \ne \emptyset \).
\end{remark}

\begin{exercise} % даже задача
  Сходимость в \( D(\Real) \)
  (пространство пробных функций из курса
  матанализа)
  не порождается никакой метрикой.
\end{exercise}

\begin{definition}
  Пусть \( X, Y \) "--- топологические пространства,
  тогда отображение \( f : X \to Y \)
  \emph{непрерывно в точке} \( x_0 \in X \),
  если
  \[
    \Forall{V(f(x_0))} \Exists{U(x_0)} f(U) \subset V.
  \]
  \( f \) \emph{непрерывно},
  если оно непрерывно в каждой точке \( x \in X \).
\end{definition}

\begin{exercise}
  Пусть \( X, Y \) "--- метрические пространства,
  \( f : X \to Y \), \( x_0 \in X \).
  Тогда следующие утверждения эквивалентны:
  \begin{enumerate}
    \item \( f \) непрерывно в \( x_0 \)
    \item Для произвольной последовательности
      \( \{ x_n \} \), сходящейся к \( x_0 \),
      \( f(x_n) \to f(x_0) \)
    \item \( \Forall{B(f(x_0))} \Exists{B(x_0)}
      f(B(x_0)) \subset B(f(x_0)) \)
  \end{enumerate}
\end{exercise}

\begin{theorem}
  Пусть \( (X, \tau_X) \), \( (Y, \tau_Y) \) "--- топологические пространства,
  \( f : X \to Y \). Тогда следующие утверждения
  эквивалентны:
  \begin{enumerate}
    \item \( f \) "--- непрерывна
    \item Если \( G \) открыто в \( Y \), то
      \( f^{-1}(G) \) открыто в \( X \)
    \item Если \( F \) замкнуто в \( Y \), то
      \( f^{-1}(F) \) замкнуто в \( X \)
  \end{enumerate}
\end{theorem}
\begin{itemproof}
  \item[\( (1) \To (2) \)] Дано: для произвольного \( x \in X \)
    \( \Forall{V(f(x))} \Exists{U(x)} f(U) \subset V \).
    Пусть  \( G \in \tau_Y \), тогда для произвольного \( x \in f^{-1}(G) \)
    \( G \) "--- окрестность \( f(x) \), а потому
    мы можем выбрать окрестность \( U(x) \) т. ч. \( f(U) \subset G \To
    U \subset f^{-1}(G) \). Значит,
    \[
      f^{-1}(G) = \bigcup_{x \in f^{-1}(G)} U(x) \in \tau_X.
    \]

  \item[\( (2) \To (1) \)]
    Для \( V(f(x)) \) возьмём \( U = f^{-1}(V) \),
    это открытое множество, и, т. к. \( V \) "--- окрестность \( f(x) \),
    \( x \in U \). При этом, \( f(U) \subset V \) 
    (равенство может не достигаться).

  \item[\( (2) \otto (3) \)]
    Заметим: для произвольного отображения
    \( f^{-1}(Y \setminus F) = X \setminus f^{-1}(F) \).
\end{itemproof}

\begin{definition}
  Пусть \( X \), \( Y \) "--- топологические пространства,
  \( f : X \to Y \) "--- биекция. Если \( f \) и \( f^{-1} \)
  непрерывны, то \( f \) называется \emph{гомеоморфизмом}.
  Если существует гомеоморфизм \( X \to Y \), то
  \( X \) и \( Y \) называются \emph{гомеоморфными}.
\end{definition}

\begin{definition}
  Пусть \( X \), \( Y \) "--- метрические пространства,
  \emph{изометрией} называется биекция \( f : X \to Y \) такая,
  что
  \[
    \Forall{x_1, x_2 \in X} \rho_X(x_1, x_2) = \rho_Y(f(x_1), f(x_2)).
  \]
  Если существует изометрия \( X \to Y \), то
  \( X \) и \( Y \) называются \emph{изометричными}.
\end{definition}

\begin{exercise}
  Пусть \( (X, \rho) \) "--- метрическое пространство,
  тогда \( \rho \) "--- непрерывное отображение,
  т. е. если \( x_n \to x \) и \( y_n \to y \),
  то \( \rho(x_n, y_n) \to \rho(x, y) \).
  Подсказка: \( |\rho(x, y) - \rho(a, b)| \le
  \rho(x, a) + \rho(y, b) \).
\end{exercise}

\begin{exercise}
  Пусть \( f : X \to Y \), \( g : Y \to Z \) "--- непрерывные
  отображения, тогда \( g \circ f \) "--- тоже непрерывно.
\end{exercise}

\begin{exercise}
  Сохраняется ли в топологических пространствах сходимость
  последовательности под действием непрерывного отображения?
\end{exercise}

\begin{definition}
  Топологическое пространство \( X \) называется \emph{несвязным},
  если \( X = G_1 \sqcup G_2 \), где
  \( G_1, G_2 \in \tau_X \setminus \{ \emptyset \} \).
  Если же \( X \) не является несвязным, то оно
  называется \emph{связным}. Множество \( Y \subset X \) называется
  связным или несвязным, если пространство
  \( Y \) с индуцированной топологией связно или несвязно,
  соответственно.
\end{definition}

\begin{exercise}
  Если \( f : X \to Y \) непрерывно и \( X \) связно,
  то \( f(X) \) тоже связно.
\end{exercise}

\begin{exercise}
  Пусть \( G \subset \Real^n \) "--- открытое множество, тогда
  \( G \) связно тогда и только тогда, когда \( G \)
  линейно связно. Придумайте замкнутое множество \( F \subset \Real^2 \),
  которое связно, но не является линейно связным.
\end{exercise}

\begin{exercise}
  Пусть \( X \), \( Y \) "--- топологические пространства,
  \( f : X \to Y \) непрерывно, \( A \subset X \) всюду
  плотно в \( X \); тогда \( f(A) \) плотно в \( f(X) \).
\end{exercise}

\end{document}
