\documentclass[main]{subfiles}

\begin{document}

\section{Сопряжённое пространство}

\begin{definition}
  Пусть \( E \) "--- ЛНП над полем \( F \),
  где \( F = \Real \) или \( F = \Complex \).
  \emph{Сопряжённым пространством} назовём
  \[ E^* \coloneqq \mathcal{L}(E, F), \]
  его элементы будем называть \emph{линейными функционалами}.
\end{definition}

\begin{example}
  Рассмотрим \( \Real_2^3 \) и линейную форму
  \( f(x) = a_1 x_1 + a_2 x_2 + a_3 x_3 \).
  Это "--- пример линейного функционала.
\end{example}

\begin{exercise}
  Пусть \( E \) "--- ЛНП, \( f \in E^* \), \( f \ne 0 \).
\end{exercise}

\begin{theorem}[Рисс-Фреше]
  Пусть \( H \) "--- гильбертово пространство
  (над \( \Real \) или \( \Complex \)),
  \( f \in H^* \). Тогда
  \[ \ExistsOne{x \in H} \Forall{h \in H} f(h) = (h, x). \]
\end{theorem}
\begin{proof}
  Покажем единственность. Если \( \Forall{h \in H}
  (h, x_1) = (h, x_2) \), то выбрав \( h = x_1 - x_2 \)
  получим \( (x_1 - x_2, x_1) = (x_1 - x_2, x_2) \To
  (x_1 - x_2, x_1 - x_2) = ||x_1 - x_2||^2 = 0 \To x_1 = x_2 \).

  Если \( f = 0 \), то можем положим \( x = 0 \).
  Иначе, \( Ker f = M \ne H \), и тогда
  \( H = M \oplus M^\perp \), и \( M^\perp \ne \{ 0 \} \).
  Выберем ненулевой \( x_0 \in M^\perp \). Для произвольного
  \( h \in H \) положим \( y = h - \frac{f(h)}{f(x_)} x_0 \),
  тогда \( f(y) = f(h) - \frac{f(h)}{f(x_0)} f(x_0) = 0 \),
  т. е. \( y \in \Ker f \To y \perp x_0 \).
  \[ (h, x_0) = (y + \frac{f(h)}{f(x_0)}, x_0) =
    (y, x_0) + \frac{f(h)}{f(x_0)} (x_0, x_0) =
  \frac{||x_0||^2}{f(x_0)} f(h) \To
  f(h) = (h, \frac{\overline{f(x_0)}}{||x_0||^2} x_0),
  \]
  а значит, мы можем положить \( x = \frac{\overline{f(x_0)}}{||x_0||^2} x_0 \).
\end{proof}

\begin{proof}[Координаты]
  Если \( H \) "--- сепарабельно, то мы можем выбрать ОНБ \( \{ e_n \} \),
  и \[ \Forall{h \in H} h = \sum_{n = 1}^\infty (h, e_n) e_n. \]
  Т. к. \( f \) "--- непрерывен, то \( f(S_n) \to f(h) \), т. е.
  \( f(h) = \sum (h, e_n) f(e_n) \).
  Заметим: если \( (x, y) = \sum (x, e_n) \overline{(y, e_n)} \).
\end{proof}
\begin{corollary}
  \( ||x||_H = ||f||_{H^*} \).
\end{corollary}

\begin{definition}
  Пусть \( E \) "--- ЛНП, \( \{ x_n \} \subset E \). Тогда 
  \( \{ x_n \} \) слабо сходится к \( x \in E \), если
  \( \Forall{f \in E^*} f(x_n) \to f(x) \)
\end{definition}
\begin{remark}
  Для гильбретого пространства \( H \) \( \{ x_n \} \)
  слабо сходится к \( x \), если \( \Forall{h \in H}
  (x_n, h) \to (x, h) \).
\end{remark}

  

\end{document}
