\documentclass[main]{subfiles}

\begin{document}
\section{Компактность}

\begin{theorem}\label{thm:compact}
  Если \( X \) "--- метрическое пространство,
  то следующие условия эквивалентны:
  \begin{enumerate}
    \item \( X \) "--- компакт
    \item \( X \) "--- полное и вполне ограниченное
    \item Из любой последовательности в \( X \)
      можно выделить сходящуюся подпоследовательность
    \item Всякое бесконечное множество имеет предельную точку
  \end{enumerate}
\end{theorem}
\begin{itemproof}
\item[$ 1 \To 2$]
  Сначала докажем полноту.
  Выберем произвольную фундаментальную последовательность
  \( \{ x_n \} \), рассмотрим её "<хвосты">:
  \( A_n = \{ x_n, x_{n+1}, \dots \} \).
  Очевидно, \( \{ A_n \} \) "---
  центрированная система замкнутых множеств
  (ведь \( \overline{A_{n+1}} \subset \overline{A_n} \)).
  Тогда по теореме (ЦСЗМ) \( \bigcap \overline{A_n} \ne \emptyset \),
  выберем \( x_0 \in \bigcap \overline{A_n} \); покажем,
  что это \( x_n \to x_0 \).
  Т. к. \( \{ x_n \} \) фундаментальна, 
  \( \Forall{\epsilon > 0} \Exists{N} \Forall{n, m \ge N}
  \rho(x_n, x_m) < \epsilon \).
  Тогда при фиксированном \( \epsilon \) выберем \( n = N \),
  и получим, что \( A_n \subset B(x_n, \epsilon) \To
  \overline{A_n} \subset \overline{B(x_n, \epsilon)}
  \subset \overline{B}(x_n, \epsilon) \),

  Теперь докажем вполне ограниченность \( X \).
  Очевидно, для произвольного \( \epsilon > 0 \)
  \[ X = \bigcup_{x \in X} B(x, \epsilon). \]
  Значит, это покрытие открытыми множествами,
  и мы можем выбрать конечное
  подпокрытие \( B(x_1, \epsilon), \dots, B(x_n, \epsilon) \),
  и тогда \( \{ x_1, \dots, x_n \} \) будет конечной \( \epsilon \)-сетью.
\item[$2 \To 3$]
  Зафиксируем произвольную последовательность \( \{ x_n \} \).
  С помощью того, что \( X \) ВО, выберем фундаментальную
  подпоследовательность; из полноты мы получим её сходимость.
  Для \( \epsilon > 0 \), выберем
  \( \epsilon \)-сеть \( B(z_1, \epsilon), \dots, B(z_k, \epsilon) \).
  В последовательности \( \{ x_n \} \) бесконечно много элементов,
  а потому в одном из шаров будет также находиться
  бесконечное число элементов, и они образуют подпоследовательность.
  Взяв \( \epsilon = 1 \), получим последовательность
  \( x_n^{(1)} \); из неё выделим таким же образом
  подпоследовательность \( x_n^{(2)} \) для \( \epsilon = \frac{1}{2} \),
  и продолжим этот процесс на \( n \)-ом шаге выбирая
  \( \epsilon = \frac{1}{n} \).
  Заметим: \( x_n^{(n)} \) и \( x_{n+p}^{(n+p)} \)
  лежат в одном шаре радиуса \( \frac{1}{n} \),
  а потому \( \rho(x_n^{(n)}, x_{n+p}^{(n+p)}) < \frac{2}{n} \).
  Отсюда получаем, что \( \{ x_n^{(n)} \} \) фундаментальна,
  а потому сходится.
\item[$3 \To 1$]
  Для начала, покажем, что \( X \) вполне ограниченно.
  Пусть это не так, т. е. для некоторого \( \epsilon_0 \)
  никакое конечное множество не является \( \epsilon_0 \)-сетью.
  Выберем произвольную точку \( x_1 \in X \);
  \( \{ x_1 \} \) "--- не \( \epsilon_0 \)-сеть,
  поэтому \( \Exists{x_2 \in X} B(x_1, \epsilon_0) \).
  Но \( \{ x_1, x_2 \} \) также не является \( \epsilon_0 \)-сетью,
  а потому \( \Exists{x_3 \in X} \rho(x_1, x_3), \rho(x_2, x_3) > \epsilon_0 \).
  Продолжая этот процесс, построим последовательность \( \{ x_n \} \)
  такую, что \( \Forall{i \ne j} \rho(x_i, x_j) \ge \epsilon_0 \).
  Ясно, что из неё нельзя выделить сходящуюся подпоследовательность,
  что противоречит предположению, а потому \( X \) "--- вполне ограниченно.

  Предположим теперь, что \( X \) не компактно:
  из покрытия \( \{ G_\alpha \} \) нельзя выбрать конечное
  подпокрытие. Для произвольного \( \epsilon > 0 \)
  можем выбрать конечную \( \epsilon \)-сеть,
  и тогда некоторый шар \( B_\epsilon = B(x_\epsilon, \epsilon) \)
  нельзя покрыть конечным числом множеств из \( \{ G_\alpha \} \).
  Положим \( x_n \) центром такого шара при \( \epsilon = \frac{1}{n} \).
  Тогда мы можем выбрать подпоследовательность
  \( x_{n_k} \to x_0 \). Кроме того, \( \Exists{\alpha_0} x_0 \in G_{\alpha_0} \).
  Это множетсво октрыто, а потому \( \exists B(x_0) \subset G_{\alpha_0} \).
  Очевидно, для некоторого \( k_0 \) \( B_{n_{k_0}} \subset B(x_0)
  \subset G_{\alpha_0} \), т. е. \( G_{\alpha_0} \) покрывает
  \( B_{n_{k_0}} \), что невозможно по построению.
  Значит, предположение неверно и \( X \) компактно.
\item[$3 \To 4$]
  Пусть \( M \subset X \) бесконечно, тогда
  в нём можно выбрать последовательность
  \( \{ x_n \} \), в которой элементы различны.
\item[$4 \To 3$]
\end{itemproof}

\begin{corollary}
  Пусть \( X \) "--- МП, \( M \subset X \).
  \begin{enumerate}
    \item Если \( M \) "--- компакт, то \( M \) замкнуто.
    \item Если \( X \) "--- полно, а \( M \) "---
      замкнуто и вполне ограниченно, то \( M \) компактно.
    \item Если \( X = \Real^n \), а \( M \)
      замкнуто и ограниченно, то \( M \) "--- компактно.
  \end{enumerate}
\end{corollary}

\begin{exercise}
  Если \( X \) "--- КТП, \( Y \) "--- ТП
  и \( f : X \to Y \) "--- непрерывное отображение,
  то \( f(X) \) "--- компакт.
\end{exercise}

\begin{definition}
  Если \( (X, \rho_X) \) и \( (Y, \rho_Y) \)
\end{definition}
\begin{theorem}[Кантор]
  Пусь \( X \) "--- КМП, \( f \in C(X) \),
  тогда \( f \) "--- равномерно непрерывна на \( X \).
\end{theorem}
\begin{proof}[Прямое доказательство]
  Для произвольного \( \epsilon > 0 \)
  \( \Forall {x \in X} \Exists{r(x)} B(x, r)
  \Forall{y \in B} |f(x) - f(y)| < \frac{\epsilon}{2} \).
  Выберем из открытого покрытия
  \( \{ B(x, \frac{r(x)}{2}) \} \) компактное
  подпокрытие \( B_1, \dots B_n \)
  и обозначим \[ \delta = \min_i \frac{r(x_i)}{2} > 0. \]
  Покжаем, что \( \Forall{x, y \in X} \rho(x, y) < \delta
  \To |f(x) - f(y)| < \epsilon \).
  Для некоторого \( i \) \( x \in B_i \subset B(x_i, r) \),
  и при этом \( \rho(x_i, y) \le \rho(x_i, x) + \rho(x, y)
  < \frac{r_i}{2} + \delta \le r_i \To y \in B(x_i, r) \).
  Наконец, \( |f(x) - f(y)| \le |f(x) - f(x_i)| + |f(x_i) - f(y)| <
  \frac{\epsilon}{2} + \frac{\epsilon}{2} = \epsilon \).
\end{proof}
\begin{proof}[Доказательство от противного]
  Пусть это не так, т. е.
  \( \Exists{\epsilon_0 > 0} \Forall{\delta > 0}
  \Exists{x, y, \rho(x, y) < \delta} |f(x) - f(y)| \ge \epsilon_0 \).
  Выберем последовательность таких пар \( (x_n, y_n) \)
  для \( delta_n = \frac{1}{n} \); \( \rho(x_n, y_n) \to 0 \),
  \( |f(x) - f(y)| \ge \epsilon_0 \).
  По теореме~\ref{thm:compact}, можем выбрать подопоследовательность.
  \( x_{n_k} \to x_0 \). Но тогда и \( y_{n_k} \to x_0 \),
  а т. к. \( f \) непрерывна, то \( f(x_{n_k}), f(y_{n_k}) \to f(x_0) \),
  что противоречит тому, что \( |f(x_{n_k}) - f(y_{n_k})| \ge \epsilon_0 \).
\end{proof}

\begin{definition}
  Пусть \( X \) "--- метрическое пространство, \( M \subset X \)
  называется \emph{предкомпактным множеством}, если его замыкание компактно.
\end{definition}
\begin{example}
  \( X = \Real \), \( M = (0, 1) \).
\end{example}

\begin{exercise}
  Если \( X \) "--- метрическое пространство, \( M \subset X \)
  предкомпактно, то \( M \) "--- вполне ограниченно.
  Если же \( X \) полно, а \( M \) вполне ограниченно,
  то \( X \) "--- предкомпактно.
\end{exercise}

\begin{example}
  \begin{enumerate}
    \item \( X = \Real^n \), тогда \( M \) ПК \( \oTTo \) \( M \) ограниченно.
    \item \( X = l_2 \): \( M \) "--- компакт \( \oTTo \)
      \( M \) замкнуто, ограниченно и
      \[ \Forall{\epsilon > 0} \Exists{N} \Forall{ \{ x_n \} \in M }
      \sum_{n = N + 1}^{\infty} |x_n|^2 < \epsilon. \]
    \item \( l_p \): Колмогоров, 1
  \end{enumerate}
\end{example}

\begin{definition}
  \( M \subset C(X) \) равностепенно непрерывно, если
  \( \Forall{\epsilon > 0} \Exists{\delta > 0}
  \Forall{x, y} \rho(x, y) < \delta \To \Forall{f \in M}
  |f(x) - f(y)| < \epsilon \).
\end{definition}

\begin{theorem}[Арцела"---Асколи]
  Пусть \( X \) "--- КМП.
  Тогда \( M \subset C(X) \)
  является ПК \( \oTTo \)
  \( M \) ограниченно в \( C(X) \) (равномерно ограниченно)
  и равностепенно непрерывно.
\end{theorem}
\begin{proof}
  Т. к. \( C(x) \) полно, достаточно показать эквивалентность
  вполне ограниченности и данных условий.

  Если \( M \) ВО, то
  ограниченность следует очевидным образом,
  докажем равностепенную непрерывность.
  Зафиксируем \( \epsilon > 0 \).
  Выберем конечную \( \epsilon \)-сеть
  \( \{ \phi_1, \dots, \phi_n \} \).
  Все эти функции непрерывны, т. е.
  \( \Exists{\delta_i > 0} \rho(x, y) < \delta_i \To
  |\phi_i(x) - \phi_i(y)| < \epsilon \);
  обозначим \( \delta = \min \{ \delta_1, \dots, \delta_n \} \).
  По определению \( \epsilon \)-сети,
  \( \Forall{f \in M} \Exists{i} ||f - \phi_i|| \le \epsilon \),
  т. е. \( \Forall{x \in X} |f(x) - \phi_i(x)| \le \epsilon \).
  Возьмём \( x, y \in X \) такие, что \( \rho(x, y) < \delta \),
  тогда \( |f(x) - f(y)| \le |f(x) - \phi_i(x)| +
  |\phi_i(x) - \phi_i(y)| + |\phi_i(y) - f(y)| <
  3\epsilon \), что и требовалось.

  Пусть теперь \( M \) равностепенно непрерывно и ограниченно,
  построим для произвольного \( \epsilon > 0 \) конечную \( \epsilon \)-сеть,
  но только для \( X = [a, b] \).
  По условию равностепенной непрерывности \( \Exists{\delta > 0}
  |x - y| < \delta \To \Forall{f \in M} |f(x) - f(y)| < \epsilon \).
  Кроме того, по условию ограниченности, функции из \( M \) принимают значения
  только в отрезке \( [-K, K] \).
  Покроем \( [a, b] \times [-K, K] \) сеткой с клетками размера
  \( \delta \times \epsilon \), и обозначим \( \{ \psi_k \} \)
  как множество всех функций-ломанных, принимающих значения в узлах этой сетки
  (очевидно, их конечно много). Покажем, что это \( 5\epsilon \)-сеть.
  Если \( a = x_0 \le x_1 < \dots, x_n = b \) "---
  координаты вертикальных линий нашей сетки и границы отрезка,
  то для произвольной \( f \in M \) мы можем выбрать \( \psi \) из нашего набора
  такое, что \( \Forall{k} |f(x_k) - \psi(x_k)| < \epsilon \).
  Выберем теперь произвольный \( x \in [a, b] \), для него найдётся
  \( k \) такое, что \( x \in [x_k, x_{k+1}] \). Тогда
  \[
    |f(x) - \psi(x)| \le |f(x) - f(x_k)| + |f(x_k) - \psi(x_k)| +
    |\psi(x_k) - \psi(x)|.
  \]
\end{proof}

\end{document}
