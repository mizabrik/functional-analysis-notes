\documentclass[main]{subfiles}

\begin{document}

\section{Нормированные пространства}
\begin{definition}
  Пусть \( E \) "--- ЛП (над \( \Real \) или \( \Complex \)).
  Нормой называется отображение \( ||\cdot|| : E \to \Real \)
  такое, что
  \begin{enumerate}
    \item \( ||x|| \ge 0 \), \( ||x|| = 0 \oTTo x = 0 \)
    \item \( ||\alpha x|| = |\alpha| ||x|| \) (однородность)
    \item \( ||x + y|| \le ||x|| + ||y|| \) (неравенство треугольника)
  \end{enumerate}
\end{definition}

\begin{exercise}
  ЛНП \( E, ||\cdot|| \) становится МП, если ввести метрику
  \( \rho(x, y) = ||x - y|| \)
\end{exercise}

\begin{definition}
  Пусть \( M \subset E \). Если \( M \) замкнуто относительно
  линейных операций, то \( M \) называется \emph{линейным многообразием}.
  Если \( M \) также замкнуто, то оно называется подпространством.
\end{definition}

\begin{definition}
  Для произвольного множества \( S \subset E \) определим
  \emph{линейную оболочку} \( [S] \) как множество всевосможных линейных комбинаций
  конечного числа элементов \( S \).
\end{definition}

\begin{example}
  \( E = C[a, b] \), \( M = [1, x, \dots, x^n, \dots] \) "--- многочлены.
  Тогда \( M \) "--- линейное многообразие, но не подпространство,
  т. к. \( \overline{M} = E \ne M \).
\end{example}

\begin{definition}
  \( \{ e_n \}_1^\infty \subset E \) называется базисом  в \( E \),
  если
  \[ \Forall{x \in E} \ExistsOne{ \{ \alpha_n \}_1^\infty }
  \sum_1^\infty \alpha_n e_n = x, \]
  где сумма ряда понимается как предел частичных сум
  \( S_N = \sum_{n = 1}^N \alpha_n e_n \) по норме.
\end{definition}

\begin{definition}
  Нормы \( ||\cdot||_1 \) и \( ||\cdot||_2 \) над ЛП \( E \)
  называются \emph{эквивалентными}, если
  \( \Exists{C_1, C_2 > 0} \Forall{x \in E} C_1 ||x||_2 \le ||x||_1 \le ||x||_2 \).
  \( ||\cdot||_1 \) \emph{слабее}, чем \( ||\cdot||_2 \),
  если \( x_n \to x \) по \( ||\cdot||_2 \) влечёт
  \( x_n \to x \) по \( ||\cdot||_1 \).
\end{definition}

\begin{proposition}
  Пусть \( E \) "--- ЛП, \( \dim E < \infty \). Тогда все нормы на
  \( E \) эквивалентны.
\end{proposition}
\begin{proof}
  Т. к. \( E \) конечномерно, можно выбрать базис
  \( e = \{ e_1, \dots, e_n \} \), т. е.
  \[ \Forall{x \in E} x = \sum_{k = 1}^n \xi_k e_k. \]
  Введём на \( E \) скалярное произведение, в котором
  \( e \) ортонормированн:
  \[ (x, y) = \sum_{k = 1}^n \xi_k \overline{\zeta_k}, \]
  оно порождает норму \( ||x||_2 = \sqrt{(x, x)} \).

  Выберем произвольную норму \( ||\cdot|| \) над \( E \)
  и покажем, что она эквивалентна \( ||\cdot||_2 \);
  если это так, то и две произвольные нормы эквивалентны.

  Покажем, что \( ||\cdot|| \) слабее:
  \[ ||x|| = ||\sum \xi_k e_k|| \le \sum ||\xi_k e_k|| =
  \sum |\xi_k| ||e_k|| \le \max_k ||e_k|| \sum |\xi_k| \le
  \max_k ||e_k|| \sqrt{\sum |\xi_k|^2} = C_2 ||x||_2 \]

  Покажем теперь, что \( ||\cdot||_2 \) слабее, чем \( ||\cdot|| \).
  Пусть это не так, тогда
  \( \Forall{m} \Exists{x_m} \frac{1}{m} ||x_m||_2 \ge ||x_m|| \).
  Рассмотрим последовательность 
  \( y_m = \frac{x_m}{||x_m||_2} \in S_{||\cdot||_2}(0, 1) \).
  \( S(0, 1) \) "--- компакт, а потому существует сходящаяся подпоследовательность,
  б. о. о. считаем, что это все элементы: \( ||y_m - y||_2 \to 0 \),
  откуда \( ||y_m||_2 \to ||y||_2 \). Кроме того, т. к. \( ||\cdot|| \) слабее,
  то верно и \( ||y_m - y|| \to 0 \To ||y_m|| \to ||y|| \).
  При этом, \( ||y_m|| = \frac{||x_m||}{||x_m||_2} < \frac{1}{m} \to 0 \),
  и тогда \( ||y_m|| \to 0 = ||y|| \). Значит, \( y = 0 \To ||y||_2 = 0 \ne 1 \),
  противоречие.
\end{proof}

\begin{exercise}
  \( y_m \to y \To ||y_m|| \to ||y|| \).
\end{exercise}

\begin{theorem}[Ф. Рисс]
  Пусть \( E \) "--- НП, \( \dim E = \infty \).
  Тогда единичная сфера \( S(0, 1) \) не ялвяется компактом.
  Более того, она даже не вплоне ограниченна.
\end{theorem}

\begin{lemma}[о "<почти"> перпендикуляре]
  Пусть \( E \) "--- ЛНП, \( M \subset E \) "--- подпространство,
  \( M \ne E \). Тогда \( \Forall{\epsilon > 0} \Exists{y \in S(0, 1)}
  \rho(y, M) > 1 - \epsilon \).
\end{lemma}
\begin{proof}
  Выберем призвольный \( y_0 \in E \setminus M \),
  обозначим \[ d = \rho(y_0, M) = \inf_{z \in M} ||y - z||. \]
  \( d > 0 \), т. к. \( M \) замкнуто, а потому
  \( \Forall{\epsilon > 0} \Exists{z_0 \in M} ||y_0 - z_0|| <
  d \cdot \frac{1}{1 - \epsilon} \).
  Рассмотрим \( y = \frac{y_0 - z_0}{||y_0 - z_0||} = \alpha (y_0 - z_0) \).

  Для произвольного \( z \in M \)
  \[ ||y - z|| = ||\alpha(y_0 - z_0) - z|| =
  |a| ||y_0 - \underbrace{(z_0 + \frac{z}{\alpha})}_{\in M}||
  \ge |\alpha| d > 1 - \epsilon. \]
\end{proof}

\begin{proof}[Доказательство теоремы]
  Выберем \( y_1 \in S(0, 1) \), т. к. \( \dim E = \infty \),
  \( [y_1] = M_1 \ne E \). Тогда по лемме можем выбрать
  \( y_2 \in S(0, 1) \) такой, что \( \rho(y_2, M_1) \ge 1 - \epsilon \),
  но опять же \( [y_1, y_2] = M_2 \ne E \). Продолжая этот процесс,
  получим \( \{ y_n \}_{n = 1}^\infty \subset S(0, 1) \) такие, что
  \( \rho(y_n, y_m) \ge 1 - \epsilon \), и тогда \( S(0, 1) \)
  нельзя покрыть конечной \( (1 - \epsilon) \)-сетью.
\end{proof}

\begin{exercise}
  Пусть \( E \) "--- ЛНП, \( \{ x_1, \dots, x_n \} \subset E \).
  Тогда \( [x_1, \dots, x_n] \) "--- замкнута в топологическом
  смысле \( \To \) это "--- подпространство.
  Подсказка: воспользуйтесь тем, что в конечномерном пространстве
  все нормы эквивалентны.
\end{exercise}

\end{document}
