\documentclass[main]{subfiles}

\begin{document}

\section{Самосопряжённые операторы}%12

В данном параграфе мы будем рассматривать
комплексные гильбертовы пространства.

\begin{definition}
  Пусть $H$ "--- гильбертово пространство,
  $A \in \Linears{H}$. $A$ называется
  \emph{самосопряжённым}, если $A^* = A$.
\end{definition}

\begin{exercise}
  В \( H = \ell_2(\Complex) \),
  выберем стандартный базис \( \{ e^n \} \),
  тогда оператор \( A \) задаётся бесконечной матрицей
  \( (a_{ij} \). Докажите, что
  оператор \( A \) самосопряжён \( \oTTo \)
  \( a_{ji} = \overline{a_{ij}} \).
\end{exercise}

\begin{exercise}
  Привести пример самосопряжённого (?) оператора \( A \)
  над сепарабельным гильбертовом пространством такого,
  что из его собственных векторов нельзя выбрать
  ортонормированный базис.
\end{exercise}

\begin{remark}
  Самосопряжённость подразумевает линейность и ограниченность,
  свойство $\Inner{Ax, y} = \Inner{x, Ay}$ определяет класс
  \emph{симметрических} операторов.
  Однако, теорема Хеллингера-Тёлпица утверждает,
  что симметрический линейный оператор будет также ограниченным,
  а потому и самосопряжённым.
\end{remark}

\begin{theorem}%12.1
  Пусть \( H(\Complex) \) "--- гильбертово пространство,
  \( A \) "--- ССО на \( H \). Тогда
  \begin{enumerate}
    \item \( \Forall{x \in H} \Inner{Ax, x} \in \Real \),
    \item если \( \lambda \) "--- собственное значение \( A \),
      то \( \lambda \in \Real \),
    \item если \( \lambda_1 \ne \lambda_2 \) "---
      собственные значения \( A \),
      а \( e_1 \) и \( e_2 \) "--- соответствующие им
      собственные вектора, то \( \Inner{e_1, e_2} = 0 \).
  \end{enumerate}
\end{theorem}

\begin{exercise}
  Если \( \Forall{x \in H} \Inner{Ax, x} \in \Real \),
  то \( A \) "--- ССО.
\end{exercise}

\begin{proof}~\begin{enumerate}
  \item С одной стороны, \( \Inner{Ax, x} = \Inner{x, Ax} \);
    с другой, \( \Inner{Ax, x} = \overline{\Inner{x, Ax}} \).
    Значит, \( \Inner{Ax, x} \in \Real \).
  \item Если \( \lambda \) "--- с. з., то
    \( \Exists{e \ne 0} Ae = \lambda e \).
    Тогда \( \Inner{Ae, e} = \Inner{\lambda e, e} =
    \lambda \Inner{e, e} \) "---
    вещественное число, а т. к. \( \Inner{e, e} \in \Real \),
    то и \( \lambda \in \Real \).
  \item pass
\end{enumerate}\end{proof}

Для ССО \( A \) и \( \lambda \in \Complex \)
\( A_\lambda^* = (A - \lambda I)^* = A^* - \overline{\lambda} I^* =
A - \overline{\lambda} I = A_{\overline{\lambda}} \),
и тогда по Т.11.2
\( \overline{\Img A_\lambda} \oplus
\Ker A_{\overline{\lambda}} = H \).

\begin{theorem}
  \( \overline{\Img A_\lambda} \oplus
  \Ker A_{\lambda} = H \).
\end{theorem}
\begin{proof}
  Если \( \lambda \in \Real \), то попросту
  \( \overline{\lambda} = \lambda \).
  Иначе, ни \( \lambda \),
  ни \( \overline{\lambda} \) не являются собственными
  значениями \( A \), и тогда
  \( \Ker A_\lambda = \{ 0 \} = \Ker A_{\overline\lambda} \).
\end{proof}

\begin{theorem}[критерий принадлежности числа спектру самосопряжённого оператора]%12.3
  Пусть \( H \) "--- комплексное гильбертово пространство,
  \( A \in \Linears{H} \) "--- самосопряжённый оператор.
  Тогда
  \begin{enumerate}
    \item \( \lambda \in \rho(A) \oTTo
      \Exists{m > 0} \Forall{x \in H} ||A_\lambda x|| \ge m ||x|| \)
    \item \( \lambda \in \sigma(A) \oTTo
      \Exists{ {\{ x_n \}}_{n=1}^\infty }
      \Forall{n} ||x_n|| = 1, \: \:
      ||A_\lambda x_n|| \to 0 \)
  \end{enumerate}
\end{theorem}
\begin{proof}
  Следствие слева направо в первом пункте напрямую следует из
  Т 6.1. % TODO: ссылка
  В обратну сторону та же теорема даст нам существование
  обратного оператора лишь на \( \Img A_\lambda \).
  Однако, мы уже знаем, что
  \( H = \overline{\Img A_\lambda} \oplus \Ker A_\lambda =
  \overline{\Img A_\lambda} \oplus \{ 0 \} = \overline{\Img A_\lambda} \).
  Значит, если мы покажем замкнутость \( \Img A_\lambda \),
  мы докажем, что обратный оператор к \( A_\lambda \)
  определён на всём \( H \).
  Пусть \( \{ y_m \} \) "--- последовательность из
  \( \Img A_\lambda \) и \( y_n \to y \);
  нужно доказать, что тогда \( y \in \Img A_\lambda \).
  Выберем \( \{ x_n \} \) такие, что \( A_\lambda x_n = y_n \);
  по условию,
  \[ ||x_n - x_{n+p}|| \le
    \frac1m ||A_\lambda (x_n - x_{n+p})||
    = \frac1m ||y_n - y_{n+p}||, \]
  а потому \( \{ x_n \} \) фундаментальна,
  ведь \( \{ y_n \} \) фундаментальна как сходящаяся последовательность.
  Значит, у \( \{ x_n \} \) существует предел \( x \)
  и т. к. \( A_\lambda \) "--- непрерывный оператор, то
  \( y = A_\lambda x \), то есть \( y \in \Img A_\lambda \).

  В силу того, что \( \sigma(A) \) есть дополнение к
  \( \rho(A) \), достаточно установить эквивалентость
  условия из первого пункта и отрицания условия из второго.
  Действительно, если такая константа \( m > 0\) найдётся,
  то при \( ||x_n|| = 1 \) \( ||A_\lambda x_n|| \ge m \),
  и тогда \( ||A_\lambda x_n|| \not \to 0 \).
  Наоборот, если такой константы нет,
  то для произвольного \( n \) мы сможем выбрать
  \( y_n \) такой, что \( ||A_\lambda y_n|| < \frac1n ||y_n|| \)
  (отметим, что из-за строгости неравенства
  \( ||y_n|| \) не может равняться нулю)
  В таком случае, \( x_n = \frac{y_n}{||y_n||} \)
  будет искомой последовательностью, ведь
  \[
    ||A_\lambda x_n|| = \frac{||A_\lambda y_n||}{||y_n||}
    < \frac1n \frac{||y_n||}{||y_n||} = \frac1n \to 0.
    \qedhere
  \]
\end{proof}

\begin{theorem}%12.4
  Пусть \( H \) "--- комплексное гильбертово пространство,
  \( A \in \Linears{H} \) "--- самосопряжённый оператор.
  Тогда \( \sigma(A) \subset \Real \) и
  если \( \lambda \notin \Real \), то
  \( ||R_\lambda(A)|| \le \frac{1}{|\Im \lambda|} \).
\end{theorem}

\begin{theorem}%12.5
  Пусть \( H \) "--- комплексное гильбертово пространство,
  \( A \in \Linears{H} \) "--- самосопряжённый оператор.
  Тогда \( \sigma(A) \subset [m_{-}, m_{+}] \),
  где
  \[
    m_{-} = \inf_{||x||=1} \Inner{Ax, x}, \\
    m_{+} = \sup_{||x||=1} \Inner{Ax, x},
  \]
  при чём, \( m_{-}, m_{+} \in \sigma(A) \).
  Кроме того,
  \[
    r(A) = ||A|| = \max \{ |m_{-}|, |m_{+}| \}.
  \]
\end{theorem}
\begin{proof}
  В силу предыдущей теоремы, достаточно
  рассматривать \( \lambda \in \Real \).
  Покажем, что если \( \lambda > m_+ \),
  то \( \lambda \in \rho(A) \).
  Достаточно найти \( m > 0 \) такое,
  что \( ||A_\lambda x|| \ge m ||x|| \).
  По неравенству Коши-Буняковского,
  \( ||A_\lambda x|| \cdot ||x|| \ge
  |\Inner{A_\lambda x, x}| = 
  |\Inner{A x, x} - \lambda ||x||^2| \).
  При этом, очевидно, \( \Inner{A x, x} \le ||x||^2 m_+ \),
  и, раз \( \lambda > m_+ \),
  \( |\Inner{Ax, x} - \lambda ||x||^2| =
  \lambda ||x||^2 - \Inner{Ax, x} \ge
  (\lambda - m_+) ||x||^2 \).
  Объединяя две части и сокращая на \( ||x|| \),
  мы получаем неравенство
  \( ||A_\lambda x|| \ge (\lambda - m_+) ||x|| \),
  т. е. нам подходит \( m = \lambda - m_+ \).

  По Т.7.1 \( r(A) = \lim_{n\to\infty} \sqrt[n]{||A^n||} \).
  Легко показать, что \( ||A^2|| = ||A||^2 \);
  очевидно, \( ||A^2|| \le ||A||^2 \), и, с другой стороны,
  \[ ||A x||^2 = \Inner{Ax, Ax} = \Inner{A^2x, x}
  \le ||A^2 x|| \cdot ||x||ю \]
  Поскольку существует предел, мы можем найти его
  по произвольной подпоследовательности, а потому
  \[
    r(A) = \lim_{k \to \infty} \sqrt[2^k]{||A^{2^k}||} =
    \lim_{k \to \infty} \sqrt[2^k]{||A||^{2^k}} = ||A||.
    \qedhere
  \]
\end{proof}

\begin{exercise}
  Лемма \( ||A^n|| = ||A||^n \) для ССО.
\end{exercise}

\begin{exercise}
  Докажите неравенство Коши-Буняковского для полускалярного
  произведения.
\end{exercise}

\end{document}
