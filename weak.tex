\documentclass[main]{subfiles}

\begin{document}
\section{Слабая сходимость} % 9

Наименьшая топология, сохраняющая непрерывность функционалов из $E^*$.

\begin{definition}
  Пусть $E$ "--- ЛНП (над $\Complex$ или $\Real$, вариант "--- ЛТП),
  тогда последовательность $\{ x_n \} \subset E$ \emph{слабо сходится} к
  $x \in E$, если $\Forall{f \in E^*} f(x_n) \to f(x)$.
\end{definition}

\begin{remark}
  Пусть $\{ x_n \}$ слабо сходится к $x'$ и $x''$.
  Т. к. предел числовой последовательности единственнен,
  $f(x') = f(x'')$ для произвольного $f \in E$, и по следствию
  (3) $x' = x''$, т. е. слабый предел единственнен.
\end{remark}

Заметим: $f(x_n) \to f(x)$ суть то же, что и
$F_{x_n}(f) \to F_x(f)$, т. е. слабая сходимость
в $E$ эквивалентна поточечной сходимости в $E^{**}$.

Если $||x_n - x|| \to 0$, то $x_n \weakto x$.
Обратное не верно: $E = l_2(\Real)$, тогда
$e^n \weakto 0$, но расходится по норме.

\begin{exercise}
  Доказать, что для произвольного $x \in B(0, 1)$
  можно выбрать последовательность $\{ x_n \} \subset S(0, 1)$
  такую, что $x_n \weakto x$.
\end{exercise}

\begin{exercise}
  Если $\dim E < \infty$, то сходимость по норме, слабая сходимость
  и покоординатная сходимости эквивалентны.
  Подсказка: в конечномерном пространстве все нормы эквивалентны.
\end{exercise}

\begin{theorem}[Критерий слабой сходимости]
  $x_n \weakto x \oTTo$ $\{ ||x_n|| \}$ ограниченно
  и $f(x_n) \to f(x)$ $\forall f \in S$, где
  $\overline{[S]} = E^*$.
\end{theorem}
\begin{proof}
  Изометрическое вложение $E \to E^{**}$, $x \mapsto F_x$,
  $F_x(f) = f(x)$.
  По следствию 4 ТХ-Б это "--- изометрия.
  Тогда слабая сходимость "--- суть поточечная сходимость
  $F_{x_n}$ к $F_x$, а тут мы можем применить теорему
  Банаха-Штейнгаузера (?).
\end{proof}

\begin{theorem}
  Пусть $E_1$, $E_2$ "--- БП,
  $A \in \mathcal{L}(E_1, E_2)$, $x_n \weakto x$.
  Тогда $A x_n \weakto Ax$.
\end{theorem}
\begin{proof}
  Случай $\Real$.
  Выберем $g \in E_2^*$ и рассмотрим $f = g \circ A$.
  $f$ "--- суперпозиция двух линейных и непрерывных отображений,
  а потому $f \in E_1^*$, и по определению слабой сходимости
  \[ g(A x_n) = f(x_n) \to f(x) = g(A x). \qedhere \]
\end{proof}

\begin{definition}
  \emph{Секвенциально слабым замыканием} $M \subset E$ назовём
  \[ \weakclosure{M} = \{ x \in E \mid \Exists{ \{ x_n \} \subset M } x_n \weakto x \}. \]
\end{definition}

\begin{definition}
  Последовательность $\{ x_n \} \subset E$ называется \emph{слабо фундаментальной},
  если для произвольного $f \in E^*$ $\{ f(x_n) \}$ "--- фундаментальная
  числовая последовательность. Если из слабой фундаментальности
  следует существование слабого предела, то $E$ называется
  \emph{секвнциально слабо полным}.
\end{definition}

\begin{exercise}
  Гильбертово $H$ является слабо секвенциально полным; для банахова пространства
  это не всегда верно.
\end{exercise}

% в C[a, b] oTTo ограниченна по норме и сходится поточечно

\begin{definition}
  $M \subset E$ "--- \emph{слабо секцвенциально компактно}, если
  из любой последовательности в $M$ можно выделеть
  слабо сходящуюся к элементу $M$
  подпоследовательность.
\end{definition}

\begin{theorem}[Банах-Тихонов-Алаоглу]
  В гильбертовом либо рефлексивном сепарабелтьном пространстве
  единичный шар "--- секвенциально слабый компакт.
\end{theorem}

\end{document}
