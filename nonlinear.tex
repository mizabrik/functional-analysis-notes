\documentclass[main]{subfiles}

\begin{document}

\section{Элементы нелинейного анализа}%15

%В вещественном анализе мы говорили,
%что функция дифференцируема в точке \( x_0 \), если
%\[
%  f(x) = f(x_0) + A (x - x_0) + o(x - x_0).
%\]
%С другой стороны, имелось определение производной
%как предела \( \frac{f(x) - f(x_0)}{x - x_0} \),
%и они были тесно связаны.
%Аналогичная ситуация наблюдалась в многомерном
%и комплексном анализе, но в бесконечномерном
%случае эта связь теряется.

\begin{definition}
  Пусть \( E_1 \), \( E_2 \) "--- нормированные вещественные пространства,
  \( D \) "--- область в \( E_1 \), \( F: D \to E_2 \).
  \( F \) \emph{дифференцируема (по Фреше)}
  (сильно дифференцируема)
  в точке \( x_0 \in D \),
  если
  \[
    \Exists{A \in \Linears{E_1, E_2}}
    \lim_{h \to 0} \frac{||F(x_0 + h) - F(x_0) - Ah||}{||h||} = 0.
  \]
  Тогда \( A \) называется \emph{производной (Фреше)}
  (сильной производной)
  функции \( F \) в точке \( x_0 \) и обозначается как \( F'(x_0) \).
  Для конкретного \( h \in E_1 \)
  \( Ah \) называется \emph{дифференциалом (Фреше)}
  (сильным дифференциалом)
  и обозначается как \( dF(x_0, h) \).
\end{definition}

\begin{definition}
  Пусть \( E_1 \), \( E_2 \) "--- нормированные вещественные пространства,
  \( D \) "--- область в \( E_1 \), \( F: D \to E_2 \).
  \emph{Дифференциалом Гато} (слабым дифференциалом)
  называется
  \[
    DF(x_0, h) = \frac{d}{dt} F(x_0 + t h) \bigr|_{t=0} =
    \lim_{t \to 0} \frac{F(x_0 + th) - F(x_0)}{t}.
  \]
  Если \( DF(x_0, \cdot) \in \Linears{E_1, E_2} \),
  то \( F \) \emph{дифференцируема по Гато} в точке \( x_0 \)
  и оператор \( DF(x_0, \cdot) \) называется
  \emph{производной Гато} (слабой производной)
  функции \( F \) в точке \( x_0 \).
\end{definition}

\begin{example}
  Пусть \( E_1 = E_2 = \Real \)
  и
  \[
    F(x, y) = \begin{cases}
      1, & x > 0, \: y = x^2 \\
      0, & \text{иначе}
    \end{cases}.
  \]
  Тогда \( DF(0, h) = 0 \) для любого \( h \in \Real^2 \),
  но \( F \) не имеет дифференцируема по Фреше.
\end{example}

\begin{exercise}
  Из дифференцируемости по Фреше следует дифференцируемость по
  Гато, и при этом обе производные совпадают.
\end{exercise}

\begin{exercise}~
  \begin{enumerate}
    \item \( F(x) \equiv const \To \Forall{x_0} F'(x_0) = 0 \)
    \item \( F \in \Linears{E_1, E_2} \To
      \Forall{x_0} F'(x_0) = F \)
    \item \( (\alpha_1 F_1 + \alpha_2 F_2)' =
      \alpha_1 F_1' + \alpha_2 F_2' \)
    \item \( (G \circ F)'(x_0) = G'(F(x_0)) \circ F'(x_0) \)
  \end{enumerate}
\end{exercise}

\begin{proposition*}
  Пусть \( D \) "--- область в \( \Real \) и
  отображение \( f : D \to \Real \)
  дифференцируемо в точке \( x_0 \in D \) по Фреше.
  Тогда \( f \) дифференцируема
  (в обычном смысле), и её производная
  (в обычном смысле)
  равна \( dF(x_0, 1) \).
\end{proposition*}
\begin{proof}
  В данном случае определение дифференцируемости
  по Фреше выглядит как
  \[
    \lim_{h \to 0} \frac{|f(x_0 + h) - f(x_0) - dF(x_0, h)|}{|h|} = 
    \lim_{h \to 0} \left| \frac{f(x_0 + h) - f(x_0) - dF(x_0, h)}{h} \right| = 0.
  \]
  При этом, \( dF(x_0, h) = dF(x_0, h \cdot 1) = h dF(x_0, 1) \),
  а потому
  \[
    \lim_{h \to 0} \left|
    \frac{f(x_0 + h) - f(x_0)}{h} - dF(x_0, 1)
    \right| = 0 \oTTo
    \lim_{h \to 0} \frac{f(x_0 + h) - f(x_0)}{h} = dF(x_0, 1).
    \qedhere
  \]
\end{proof}

\begin{theorem}[об оценке конечных приращений]
  Пусть \( E_1 \), \( E_2 \) "--- вещественные
  нормированные пространства,
  \( D \subset E_1 \) "--- выпуклая область
  и \( F : D \to E_2 \) дифференцируема на \( D \)
  (по Фреше).
  Тогда для произвольных \( x_0, x_1 \in D \)
  \[
    ||F(x_1) - F(x_0)|| \le \sup_{\xi \in (x_0, x_1)} ||F'(\xi)|| \cdot ||x_1 - x_0||,
  \]
  где \( (x_0, x_1) = \{ x_0 + t (x_1 - x_0) \mid t \in (0, 1) \} \).
\end{theorem}
\begin{proof}
  Для \( t \in [0, 1] \) определим
  \( x_t = x_0 + t (x_1 - x_0) \) и
  для произвольного \( f \in E_2^* \)
  рассмотрим композицию
  \( \phi_f = f \circ F \circ x_{\boldsymbol{\cdot}} \)
  (\( \phi_f : (0, 1) \to \Real \)).
  \( f \) дифференцируем как любой линейный функционал,
  \( F \) дифференцируема по условию
  и \( x_t'(s) : h \mapsto h (x_1 - x_0) \).
  Значит, дифференцируема и \( \phi_f \),
  при том
  \[
    d\phi_f(s, h) =
    f([F'(x_s)] (h (x_1 - x_0))).
  \]
  Поскольку \( \phi_f \) "--- просто
  вещественная функция,
  к ней можно применить теорему Лагранжа о среднем:
  \( \phi_f(1) - \phi_f(0) = d \phi_f(s, 1) \)
  для некоторого \( s \in (0, 1) \).
  По второму следствию из теоремы Хана"--~Банаха
  найдётся \( f \in E_2^* \) такой,
  что \( ||f|| = 1 \) и \( f(F x_1 - F x_0) =
  ||F x_1 - F x_0|| \);
  тогда
  \begin{align}
    ||F x_1 - F x_0|| &= |f(F x_1 - F x_0)| =
    |\phi_f(1) - \phi(0)| =
    |d \phi_f (s, 1)| = 
    |f([F'(x_s)](x_1 - x_0))| \\ 
    &\le ||f|| \cdot ||F'(x_s)|| \cdot ||x_1 - x_0|| \le
    \sup_{\xi \in (x_0, x_1)} |F'(\xi)| \cdot ||x_1 - x_0||.
    \qedhere
  \end{align}
\end{proof}

Существенную роль для приложений играют
теоремы о неподвижной точке;
мы уже доказали теорему Банаха о
неподвижной точке сжимающего отображения
и рассматривали в задании
её усиление для компактных пространств.
Другим классическим результатом
является теорема Брауэра о неподвижной точке.

\begin{theorem*}[Брауэр, 1910]
  Пусть \( B \) "--- замкнутый шар в \( \Real^k \),
  \( f : B \to B \) "--- непрерывное отображение.
  Тогда у \( f \) найдётся неподвижная точка.
\end{theorem*}

Этот результат также обобщается
и на бесконечномерный случай,
но для этого нам вновь требуется
понятие компактности.

\begin{definition}
  Отображение (не обязательно линейное)
  \( F : E_1 \to E_2 \)
  называется компактным, если
  \( F \) непрерывно и для любого ограниченного
  множества \( A \subset E_1 \)
  \( F(A) \) предкомпактно в \( E_2 \).
\end{definition}

\begin{theorem}[Шаудер, б/д]
  Пусть \( E \) "--- вещественное банахово пространство,
  \( D \subset E \) "---
  замкнутое, выпуклое и ограниченное множество,
  \( F : D \to D \) "--- компактное отображение.
  Тогда у \( F \) существует неподвижная точка.
\end{theorem}

\end{document}
