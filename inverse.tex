\documentclass[main]{subfiles}

\begin{document}
\section{Обратный оператор}

\begin{definition}
  Пусть \( E_1 \), \( E_2 \) "--- нормированные пространства,
  \( A : E_1 \to E_2 \); тогда на \( \Im A \) определён обратный
  оператор \( A^{-1} \), если \( \Forall{y \in \Im A}
  \ExistsOne{x \in E_1} Ax = y \).
\end{definition}

Правый обратный оператор: \( B : \Im A \to E_1 \) является
правым обратным к \( A \), если \( \Forall{y \in \Im A}
A(By) = y \) (существует решение для \( Ax = y \)).
Левый обратный оператор: \( C : \Im A \to E_1 \) является
левым обратным к \( A \), если \( \Forall{y \in \Im A}
C(Ay) = y \) (решение для \( Ax = y \) единственно).

\( AB = I \), \( CA = I \) \( \To
C = C I = C (AB) = (CA) B = I B = B \).

конечномерные

\( Ker A = \{ 0 \} \oTTo \exists A^{-1} \).

\begin{theorem}
  Пусть \( E_1, E_2 \) "--- НП, \( A \in \mathcal{L}(E_1, E_2) \).
  Тогда \( \exists A^{-1} \in \mathcal{L}(\Im A, E_1) \oTTo
  \Exists{m > 0} \Forall{x} ||Ax|| \ge m ||x|| \).
\end{theorem}
\begin{itemproof}
\item [$\To$] \( \Exists{M} ||A^{-1} y|| \le M||y|| \),
  \( ||x|| \le M ||Ax|| \), т. е. \( m = \frac{1}{m} \).
\item[$\oT$] \( \Exists{m} ||Ax|| \ge m ||x|| \).
  Если \( x \ne 0 \), то \( Ax \ne 0 \To \Ker A = \{ 0 \} \To
  \exists A^{-1} \), и \( ||y|| \ge m ||A^{-1} y|| \To
  ||A^{-1}|| \le \frac{1}{m} \).
\end{itemproof}

\begin{theorem}
  Пусть \( E \) "--- БП, \( A \in \mathcal{L}(E) \),
  \( ||A|| < 1 \). Тогда \( \exists (I + A)^{-1} \in \mathcal{L}(E) \).
\end{theorem}
\begin{proof}
  Идея: обобщим формулу для вещественных чисел:
  \[ \frac{1}{1 + a} = 1 - a + a^2 - \dots \]
  Итак, рассмотрим \( S_n = I - A + A^2 - \dots + (-1)^n A^n \),
  покажем что \( S_n \to S \) и \( S = (I + A)^{-1} \).

  \[ ||S_{n + p} - S_n|| = ||\sum_{k = n + 1}^{n + p} A^k|| \le
    \sum_{k = n + 1}^{n + p} ||A^k|| \le \sum_{k = n + 1}^{n + p} ||A||^k
  \To 0, \]
  т. к. сумма геометрической прогрессии сходится.
  Значит, \( { S_n} \) фундаментальна и т. к. \( \mathcal{L}(E) \)
  также банахово, \( S_n \to S \in \mathcal{L}(E) \).

  Покажем теперь, что \( S \) является и правым, и левым обратным к \( I + A \).
  Вспомним: если \( S_n \to S \), то \( \Forall{B \in \mathcal{L}(E)} BS_n \to BS,
  S_n B \to SB \). Вообще говоря, коммутативности композиции операторов нет,
  но если \( p_1 \) и \( p_2 \) "--- многочлены, то
  \( p_1(A) p_2(A) = p_2(A) p_1(A) \), и это наш случай, поэтому рассмотрим только
  один случай:
  \[ (I + A)(I - A + A^2 - \dots + (-1)^k A^k =
  I + (-1)^k A^{n + 1} \to I, \]
  т. е. \( S \) "--- действительно обратный оператор к \( I + A \).
\end{proof}
\begin{corollary}
  \[ ||(I + A)^{-1} - S_n|| = ||\sum_{k = n + 1}^\infty A^k|| \le
  \sum_{k = n+1}^\infty ||A||^k = \frac{||A||^k}{1 - ||A||}. \]
\end{corollary}

\begin{remark}
  Мы показали, что ряд Неймана \( \sum (-1)^k A^k \) соходится при
  \( ||A|| < 1 \). На самом деле, можно (и нужно для выполнения задания)
  показать, что ряд \( \sum A^k \) сходится \( \oTTo \)
  \( \Exists{k} ||A^k|| < 1 \).
\end{remark}

\begin{example}
  \( E = C[0, 1] \), \( (Af)(x) = \int\limits_0^x f(t) dt \);
  \( ||A^n|| = \frac{1}{n!} \), поэтому уточнение существенно.
\end{example}

\begin{theorem}
  Пусть \( E_1 \) "--- БП, \( E_2 \) "--- НП,
  \( A \in \mathcal{L}(E_1, E_2) \),
  \( A^{-1} \in \mathcal{L}E_2, E_1) \),
  \( \delta A \in \mathcal{L}(E_1, E_2) \),
  \( ||\delta A|| < ||A^{-1}||^{-1} \).
  Тогда \( \exists {(A + \delta A)}^{-1} \in \mathcal{L}(E_2) \).
\end{theorem}

\begin{theorem}[Банах, об обратном операторе]
  Пусть \( E_1 \), \( E_2 \) "--- банаховы пространства,
  \( A \in \mathcal{L}(E_1, E_2) \) "--- биекция.
  Тогда \( A^{-1} \in \mathcal{L}(E_2, E_1) \).
\end{theorem}

\end{document}
